\section{Usage Scenarios}
\subsection{Experimentation}
\subsubsection{Creation of Non-Realtime Multiplex}
The creation of a ETI file containing two programmes, one DAB and one \dabplus{} is covered in
section \ref{sec-files}.

\subsubsection{Modulation of ETI for Offline Processing}
The ETI file generated before can then be used with ODR-DabMod to generate a file containing I/Q
samples. Here, we must chose between using the command line or the configuration file. For a very
simple example, using the command line makes sense, but for more advanced features it is preferable
to use a configuration file. For illustration, we will present both.

To modulate the file \texttt{myfirst.eti} into \texttt{myfirst.iq}, with the default options, the
command is simply

\begin{lstlisting}
odr-dabmod myfirst.eti -f myfirst.iq
\end{lstlisting}

This will create a file containing 16-bit interleaved I/Q at $2048000$ samples per second. The
transmission mode is defined by the ETI file.

The equivalent configuration file would be
\begin{lstlisting}
[input]
transport=file
source=myfirst.eti

[output]
output=file

[fileoutput]
filename=myfirst.iq
\end{lstlisting}

This is a very minimal file that defines only the necessary settings equivalent to the above command
line options. The configuration file however supports more options that the command line, and
becomes easier to manager once the set becomes more complex. It is best to use the example
configuration availble in the \texttt{doc/} folder.

\subsection{Interfacing Hardware Devices}
\subsubsection{Ettus USRP}
ODR-DabMod integrates support for the UHD library that can interface with all USRP devices from
Ettus. The following configuration file illustrates how to send the \texttt{myfirst.eti} over a USRP
B200 on channel 13C:

\begin{lstlisting}
[remotecontrol]
telnet=1
telnetport=2121

[input]
transport=file
source=myfirst.eti
loop=1

[modulator]
gainmode=2
digital_gain=0.8

[firfilter]
enabled=1
filtertapsfile=simple_taps.txt

[output]
output=uhd

[uhdoutput]
master_clock_rate=32768000
type=b200
txgain=40
channel=13C
\end{lstlisting}

This example also shows more options that the example for the file output:

\begin{itemize}
    \item \texttt{remotecontrol telnet=1} enables the Telnet server that can be used to set
        parameters while the modulator is running.
    \item \texttt{loop=1} rewinds the input file when the end is reached. The same ETI file will be
        transmitted over and over.
    \item \texttt{gainmode=2} sets the GainMode to VAR, which reduces overshoots in the output.
    \item \texttt{digital\_gain=0.8} reduces the output sample deviation, to reduce compression in
        the USRP.
    \item \texttt{firfilter enabled=1} enables an additional FIR filter to improve the spectrum
        mask.
    \item \texttt{master\_clock\_rate=32768000} sets the USRP internal clock to a multiple of
        $2048000$, which is required if we want to use the native DAB sample rate.
    \item \texttt{txgain=40} Sets the analog transmit gain of the USRP to 40dB, which is specific to
        the B200.
\end{itemize}

Some of these options are not necessary for the system to work, but they improve the performance.

\subsubsection{Other Hardware}

\subsection{Audio sources}
\subsubsection{Using Existing Web-Streams}
\subsubsection{Encoders at Programme Originator Studios}

\subsection{Single-Frequency Networks}
