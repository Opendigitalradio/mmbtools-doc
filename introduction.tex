\section{Introduction}
This is the official documentation for the \mmbtools. These tools can be used to
experiment with DAB modulation, learn the techniques behind it and setup a DAB
or \dabplus transmitter.

\section{Purpose}
The different programs that are part of the \mmbtools each have their own
documentation regarding command-line options and configuration settings, and the
opendigitalradio.org wiki\footnoteurl{http://opendigitalradio.org}
contains many explanations and pointers, but there is
no single source of documentation available for the whole tool-set.

This document aims to solve this, by first outlining general concepts,
presenting different usage scenarios and detailing a complete transmission
setup.

With this document in hand, you should be able to understand all elements
composing a \mmbtools transmission chain, and how to set one up.

\section{Presentation of the Tools}
\subsection{Origins}
In 2002, Communications Research Centre Canada\footnoteurl{http://crc.ca}
started developing a DAB multiplexer. This effort evolved through the years, and
was published later\sidenote{when?} as \mbox{CRC-DabMux} under the GPL
open-source licence.

CRC also developed a DAB modulator, called \mbox{CRC-DABMOD}, which could create
baseband I/Q samples from an ETI file. This I/Q data could then be set to
a hardware device using another tool. For the Ettus USRPs, a ``wave player''
script was necessary to interface to GNURadio. Only DAB Transmission Mode 2 was
supported. \mbox{CRC-DABMOD} was also released under the GPL\sidenote{when?}.

As encoders, toolame could be used for DAB, and CRC developed a closed-source
\mbox{CRC-DABPLUS} \dabplus encoder.

These three CRC-~tools, and some additional services available on the now
unreachable website\footnote{There are some snapshots of the website available
    on \url{http://archive.org}.} \url{http://mmbtools.crc.ca} were
part of the \mbox{CRC-mmbTools}. These tools made it possible to set up the
first DAB transmission experiments.

In 2012, these tools received experimental support for single-frequency
networks, a functionality that has been developed by Matthias P. Braendli during
his Master's thesis\footnote{The corresponding report is available at
    \url{http://mpb.li/report.pdf}}.
Because SFNs only make sense in TM 1, CRC subsequently released a patch to
\mbox{CRC-DABMOD} that enabled all four transmission modes.

At that point, involvement from CRC started to decline. The SFN patch was
finally never included in the \mbox{CRC-mmbTools}, and as time passed by, the
de-facto fork on \url{http://mpb.li} was receiving more and more features.
Having two different programs with the same name made things complicated, and
the tools were officially forked with the approval of CRC in Feb 2014, and given
the new name \mbox{ODR-mmbTools}. They are now developed by the Opendigitalradio
association.

In April 2014, the official \mbox{CRC-mmbTools} website went offline, and it has
become very difficult, if not impossible to acquire licences for the
\mbox{CRC-DABPLUS} encoder. Luckily there is an open-source replacement
available, which was part of Google's Android sources. This encoder has been
extended with the necessary \dabplus{}-specific requirements (960-transform,
error correction, framing, etc.), and now exists under the name
\mbox{fdk-aac-dabplus}.

\subsection{Included Tools}
The \mmbtools contain the tools \mbox{ODR-DabMux}, \mbox{ODR-DabMod},
\mbox{toolame-dab}, \mbox{fdk-aac-dabplus}, and other scripts, bits and pieces
that are useful for the setup of a transmission chain.

\subsubsection{ODR-DabMux}
ODR-DabMux implements a multiplexer that is conforming to the DAB
standard~\cite{etsidab}.

\subsubsection{ODR-DabMod}
\subsubsection{toolame-dab}
\subsubsection{fdk-aac-dabplus}
\subsubsection{mmbtools-aux}



% vim: spl=en spell tw=80 et
