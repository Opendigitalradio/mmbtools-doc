% LICENSE: see LICENCE
\section{A Production Broadcast Setup}
\subsection{Outline}
We have now seen all necessary elements for a complete broadcast chain, and we
will now consider what is necessary to use these elements in a 24/7 production
environment. At this point, many previously considered topics come together to
form a reliable system.

First, let us outline what our desired setup shall include:

\begin{itemize}
    \item We want to transmit about a dozen programmes using a single
        transmission site, i.e.~no SFN;
    \item Some audio sources are web-streams, some are using remote ODR-AudioEnc
        encoders;
    \item One machine is used for audio encoding web-streams, multiplexing and
        modulation;
    \item ODR-AudioEnc instances will connect over EDI to ODR-DabMux,
        ODR-DabMod will use EDI to connecto to ODR-DabMux;
    \item All audio encoders will insert PAD with DLS, and optionally slideshow;
    \item We are transmitting using a USRP B200, driving a power amplifier;
    \item We enable both telnet and zmq remote control interfaces for management
        purposes;
    \item The power amplifier will be driven linearly, no digital-predistortion
        is used;
    \item We must respect the spectrum mask given by the broadcast license;
    \item The setup must be resilient to program failure and restart them
        automatically, also informing the operator;
    \item We use munin to monitor the operation of the system.
\end{itemize}

This skips over planning considerations like choice of site location, antenna
diagrams, appropriate transmit power or regulation aspects, as we assume these
topics are were already taken care of.
With the outline set, we will now go through a list of steps that will lead to a
functional and reliable broadcast setup.


\subsection{Setup steps}
\paragraph{Choice of computer}
First, a suitable computer has to be chosen for running the tools. As this needs
to be as reliable as possible, it is preferable to chose a server designed for
reliability. Because we are driving a USRP over USB, it is essential to have a
good USB controller on the motherboard. Sadly, there is no easy way to verify
this besides actually testing it. See section~\ref{usrp_b200} for more details.
Redundant power supplies and the ability to use two hard drives in a RAID are
also useful to have.

\paragraph{Operating System}
The operating system needs to be installed next. All the dependencies for the
tools need to be installed, as well as the additional tools needed for the
system: supervisord for process supervision, Munin for monitoring, logging and
logrotate configuration, proper NTP setup, configuring real-time scheduling and
additional topics discussed in section~\ref{systemenvironment}.
If you need to prepare remote encoders, this has to be done for all the machines
you will use.

\paragraph{Installation of the Tools}
The tools themselves need to be installed, according to instructions in the
\texttt{README} and \texttt{INSTALL} files of the repositories. Then you need to
prepare the configuration for the tools, using the examples and this guide.
For every programme, create a folder for the slideshow images and gather the
slides, and prepare the interfaces for DLS text.
Write the supervisord configuration files that are used to launch all
ODR-AudioEnc and ODR-PadEnc processes. Write the multiplex configuration, with
all the entries for your programmes and an appropriate supervisor configuration.
Setup ODR-DabMux munin monitoring as desired.

\paragraph{Verification of Multiplexer}
At this point you should already be able to launch the configured tools and
verify that they start, connect properly and stay running.
You can simulate process failures by killing any of the tools; the supervisor
should restart it.
You could use etisnoop and other ETI analysis tools to verify that your
multiplex is valid, or listen in on the programmes by using netcat piped into
dablin.\footnote{\texttt{nc MUX 9200 | dablin\_gtk} should work, assuming
your ODR-DabMux serves ETI over TCP on port $9200$. Replace \texttt{MUX} by the
multiplexer IP address.
See \url{http://github.com/Opendigitalradio/dablin} for information about
dablin.}
Also check that logging and munin monitoring works.

\paragraph{ODR-DabMod Configuration}
Next configure ODR-DabMod. For improved spectrum performance, configure it with
FIR filter enabled, OFDM symbol windowing enabled (if available), with the
frequency given in your license, and start with a digital gain of $0.5$ and a TX
gain of $60$dB. If you have a disciplined 10MHz clock reference or a GPSDO,
configure accordingly. This will ensure the modulator runs at the same rate as
the rest of the transmission chain whose rate is in turn related to NTP.

\paragraph{Generate Exciter Signal}
Prepare the ODR-DabMod supervisor configuration. Connect the USRP to a spectrum
analyser and launch the modulator. Before connecting the power amplifier, make
sure to have a good spectrum at the USRP TX port, and use the remote control
interface to modify TX gain and digital gain to see what RF power you can
generate given the spectrum mask you want to achieve. Placing a DAB receiver
next to your setup, you should also be able to verify that reception is
possible, audio is present and that the DLS and slides are properly transmitted.
Ideally, let this setup run for a couple of days and check for the absence of
underruns. This step proves you can generate a valid exciter signal with good
characteristics.

\paragraph{Connect Power Amplifier}
After stopping the transmission, connect the USRP to a Band III
filter\footnote{For example, a filter with similar characteristics as the
Mini-Circuits RBP-220W+.} to suppress harmonics, connect to the power amplifier.
Using suitable attenuation, connect the amplifier output to your spectrum
analyser.
Configure a low TX gain of $30$dB and a digital gain of $0.5$, and power up.
Again do some experimentation with both TX gain and digital gain to find the
optimal settings, now with the amplifier. Let the amplifier warm up to
operational temperature before reaching conclusions. If your amplifier has a
monitoring interface, make sure it works and integrate it into your setup.

\paragraph{Tune RF Settings}
Also experiment with settings that have an impact on the spectrum performance:
OFDM Symbol Windowing and the FIR Filter settings.
If you have measurement equipment that can demodulate and measure MER, make sure
it is within bounds, ideally better than $25$dB.\sidenote{Justify this value.}
You can trade-off MER against peak-to-average power ratio using the
\verb+normalise_variance+ and CFR settings.

\paragraph{Insert Mask Filter}
The final measurements before installation needs to be done with the mask
filter connected after the power amplifier, to ensure that the spectrum mask is
satisfied. The mask filter also needs some warm up time.
It is also advisable to use a vector-network analyser to check the
mask filter's S11 and S12 parameters.

\paragraph{Final Setup}
Finally, set up the system at your transmission site, power up to nominal power,
do coverage measurements and compare them to the simulations.
By now, you will also have to deploy all the remote encoders at the programme
originators' studios.

\paragraph{Maintenance and Monitoring}
Running a multiplex is unlikely to be a ``set up and ignore'' project. Usually
you will have to do many kinds of interventions, because of changes in your
multiplex composition, requests from programmes you are carrying (e.g.~change of
web-stream URL, replacement of slides), or notify them in case of audio issues;
equipment failure due to weather conditions requiring replacement; regular
system updates that should made with low impact; changes of configuration
related to announcements or service linking; modification of RF settings due to
aging of RF components or due to seasonal thermal changes. All these are
inherent to operating a broadcast infrastructure and create maintenance work
that needs to be planned for.

% vim: spl=en spell tw=80 et
